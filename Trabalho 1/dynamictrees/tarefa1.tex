\documentclass[10pt]{article}

\oddsidemargin -.25in \evensidemargin -.25in \topmargin .05in
\textheight 52pc 
\textwidth 40pc   %6.8
\headheight 0in \headsep 0in     %  avoid extra space for header
\parindent 0.2in
\usepackage{url}

\title{Image Analysis \\ (MO445/MC940)} 
\author{Prof. Alexandre Xavier Falc\~ao} 
\date{\today}

\begin{document}

\maketitle 

Como primeira tarefa, leiam o artigo dado sobre o método
\textit{Dynamic Trees} e implementem um código em C que recebe uma
imagem de entrada, um arquivo de sementes rotuladas e gera na saída o
resultado da segmentação em uma imagem de rótulos. O artigo apresenta
seis variantes do método com funções distintas de custo de aresta. A
tarefa consiste em comparar os resultados desses variantes em relação
ao resultado gerado no programa MISe
(\url{https://github.com/LIDS-UNICAMP/MISe}), que usa um dos variantes
como referência. Use as mesmas medidas usadas no artigo para
comparação entre os variantes. O MISe pode ser usado para visualizar
os resultados e escolher novas sementes. Note que, as sementes foram
escolhidas para um dos variantes, então pode ser necessário nova
escolha dependendo do variante. Na pasta também existe um exemplo do
método watershed, que também deve ser usado como referência na
comparação.

Para as imagens de rótulos, o fundo tem sempre rótulo 0 e os objetos
têm rótulos de 1 a n, dependendo do número n de objetos. Na pasta data
vocês irão encontrar os exemplos de imagem original, imagem de rótulos
gerada pelo MISe e arquivos de sementes rotuladas para testarem o
programa. Usem a biblioteca da ift
(https://github.com/LIDS-UNICAMP/ift) para implementar a tarefa,
convertendo as imagens coloridas de RGB para o espaço Lab -- e.g.,
iftImageToMImage(imagem,Lab\_CSPACE).

A entrega da tarefa deverá ser no dia 16/09. Enviem para
afalcao@unicamp.br um arquivo Tarefa1-SeuRA.tar.bz2, que deve conter o
seu código e o relatório apenas. O relatório deve apresentar os
resultados quantitativos e qualitativos de cada variante.

\end{document}











